% Harish's Curriculum vit� (v5.0 / 20061113)
%
% This LateX document requires Mikl�s Cs�r�s' r�sum� class; the latest
% version of which ought be available at the URI below:
% http://www.iro.umontreal.ca/~csuros/latex.html
%
% Once you've fetched the class file and dropped it into the same
% directory, you can edit this vitae to fit your needs, and ``TeX it''
% using the command: 
% latex vitae.tex && dvips -o vitae.ps vitae.dvi && ps2pdf vitae.ps

% Set the document style
\documentclass{resume}

% Define additional packages used
\usepackage[usenames]{color}

% Set constants and new commands
\renewcommand{\categoryfont}{\sc\small}
\setlength{\oddsidemargin}{1in}
\setlength{\marginparwidth}{1in}
\addtolength{\marginparwidth}{-\marginparsep}
\setlength{\evensidemargin}{\oddsidemargin}
\setlength{\textwidth}{\paperwidth}
\addtolength{\textwidth}{-2in}
\addtolength{\textwidth}{-2\oddsidemargin}
\addtolength{\textwidth}{\marginparwidth}
\addtolength{\textwidth}{\marginparsep}
\setlength{\topmargin}{-0.5in}
\renewcommand{\labelcitem}{$\circ$}
\renewcommand{\labelitemi}{$\cdot$}
\newcommand{\first}{$1^{\mbox{\scriptsize st}}$\ }
\newcommand{\second}{$2^{\mbox{\scriptsize nd}}$\ }
\newcommand{\third}{$3^{\mbox{\scriptsize rd}}$\ }

% Author
\author{$\quad\quad$Harish Narayanan} %Gross centering hack

% Contact Information
% Official
\address{
    {University of Michigan\\
      Computational Mechanics Laboratory\\
      Department of Mechanical Engineering\\
      2250 G. G. Brown, 2350 Hayward\\
      Ann Arbor, MI 48109--2122\\
      Lab Phone: +1 (734) 936--2925}}
% Personal
        {
            350 Thompson Street, Apt \#108\\
            Ann Arbor, MI 48104--2239\\
            Phone: \small{+}1 (734) 262--1010\\[12 pt]
            \mbox{\small\tt hnarayan@umich.edu}\\
            \mbox{\small\tt http://umich.edu/$\sim$hnarayan/}}
        
% Begin the document
\begin{document}

% Make the header
\maketitle

% Long term goal
\begin{category}{Career\\ Goal}
  \citemnobullet {\em To pursue a career in academia as a mechanics
    educator at a distinguished institution of higher learning.}
\end {category}

% Education
\begin{category}{Education}
  \citem{University of Michigan}, Ann Arbor, MI \hspace{4mm} (2002 -- Current)\\
  $\cdot$ Ph.D. in Mechanical Engineering and Scientific Computing,
  July 2007 {\em (Expected)}\\
  $\quad{}\quad{}$ Dissertation topic: The mechanics and physics of
  biological growth\\
  $\quad{}\quad{}$ Dissertation chair: Prof. Krishnakumar
  R. Garikipati\\[3 pt]
  $\cdot$ M.S. in Mathematics, December 2006\\[3 pt]
  $\cdot$ M.S.E. in Mechanical Engineering, December 2003
  \citem{University of Madras}, Madras, India \hspace{9mm} (1998 -- 2002)\\
  $\cdot$ B.E. in Mechanical Engineering, July 2002  {\em (First class
    with distinction)}
\end{category}

% Academic honours
\begin {category}{Academic Honours}
  \citembullet Received the {\em Sir C. P. Ramaswamy Aiyar Endowment
    Scholarship} from the University of Madras in 2001 -- 2002 for
  excellent academic performance at the undergraduate level.
  \citembullet Received a {\em Certificate of Merit} for outstanding
academic work throughout $12^{\mathrm{th}}$ grade, including securing
the {\em First Rank in Physics}, AISSCE 1998 (C.B.S.E.
$12^{\mathrm{th}}$).
\end{category}

% Intellectual things I find interesting
\begin {category} {Academic Interests}
  \citembullet Classical and modern field theories of mechanics,
  Multi-physics phenomena
  \citembullet Analysis of numerical methods, Advanced finite element
  methods
  \citembullet Large scale, high performance computing 
\end{category}

% Courses covered in graduate school
\begin{category}{Graduate Coursework}
  \citembullet Theoretical mechanics, Continuum mechanics, Solid and
  structural mechanics, Theory of elasticity, Mechanics of polymers,
  Differential equations in mechanics, Mechanical vibrations
  \citembullet Finite element, difference and volume methods,
  Multi-grid methods, Spectral methods, Level set methods, Numerical
  linear algebra, Complex analysis, Methods of applied analysis
  \citembullet Computational modelling of biological tissue,
  Multi-physics phenomena at micro-scales, General relativity,
  Parallel computing
\end{category}

%Research Work
\begin{category}{Research Experience}
  \citembullet January 2003 -- Present: {\em Graduate Student Research 
  Assistant} studying the mechanics of biological
  tissue growth. This doctoral thesis work is supervised by
  Prof. Krishnakumar R. Garikipati, Prof. Ellen M. Arruda and
  Prof. Karl Grosh in the Department of Mechanical Engineering and
  Prof. Trachette L. Jackson in the Department of Mathematics.
  \citembullet October 2002 -- December 2002: {\em Directed Research}
  under Prof. Krishnakumar R. Garikipati in collaboration with
  Prof. Michael Falk, working on numerical evaluation of Green's
  function solutions pertinent to defect 
  formation in anisotropic solids under stress.
  \citembullet August 2001 -- July 2002: {\em Undergraduate Senior
    Year Project} at the University of Madras on systematically
  evaluating the use of gaseous petroleum as a viable fuel for
  automobiles. This work was under the guidance of Prof. S. Sampath
  and Prof. B. S. Murthy.
\end{category}

%Teaching work
\begin{category}{Teaching Experience}
  \citembullet September 2006 -- December 2006: {\em Graduate Student
  Instructor} for {\em Mechanical Behaviour of Materials} (ME 382),
  working with Prof. Ellen M. Arruda and Prof. J. Wayne Jones.
\end{category}

%Go to next page
\newpage
\topmargin -0.2in

%Publication list
\begin{category}{Publications}
  \citembullet ``Biological Growth: Reaction, transport and
  mechanics,'' {\em H. Narayanan}, E. M. Arruda, K. Grosh,
  K. Garikipati, Biomechanics and Modeling in Mechanobiology (In
  preparation)
  \citembullet ``The continuum elastic and atomistic viewpoints on the
  formation volume and strain energy of a point defect,''
  K. Garikipati, M. Falk, M. Bouville, B. Puchala, {\em H. Narayanan},
  Journal of the Mechanics and Physics of Solids, Vol. 54 (9)
  pp. 1929--1951, September 2006
  \citembullet ``Biological remodelling: Stationary energy,
  configurational change, internal variables and dissipation,''
  K. Garikipati, J. E. Olberding, {\em H. Narayanan}, E. M. Arruda,
  K. Grosh, S. Calve, Journal of the Mechanics and Physics of Solids,
  Vol. 54 (7) pp. 1493--1515, \mbox{July 2006}
  \citembullet ``Characterization and modeling of growth and
  remodeling in tendon and soft tissue constructs,'' E. M. Arruda,
  S. Calve, K. Garikipati, K. Grosh, {\em H. Narayanan}, Proceedings
  of the IUTAM symposium on Mechanics of Biological Tissue, June
  27--July 2, 2004 (To appear)
  \citembullet ``Material forces in the context of biotissue
  remodelling,'' K. Garikipati, {\em H. Narayanan}, E. M. Arruda,
  K. Grosh, S. Calve, Mechanics of Material Forces edited by
  P. Steinmann and G. A. Maugin, Springer, 2005
  \citembullet ``A continuum treatment of growth in biological tissue:
  The coupling of mass transport and mechanics,'' K. Garikipati,
  E. M. Arruda, K. Grosh, {\em H. Narayanan}, S. Calve, Journal of the
  Mechanics and Physics of Solids, Vol. 52 (7) pp. 1595--1625, July 2004 
\end{category}

%Some recent talks
\begin{category}{Selected Talks}
  \citembullet ``Finite Element Methods in General Relativity,'' 2006
  University of Michigan Engineering Graduate Student Symposium, November 2006
  \citembullet ``Viscoelastic and Growth Mechanics in Engineered and
  Native Tendons,'' 43$^\mathrm{rd}$ Annual Technical Meeting of the
  Society of Engineering Science, August 2006
  \citembullet ``The numerical implications of fluid incompressibility
  in multiphasic modelling of soft tissue growth,'' Seventh World
  Congress on Computational Mechanics, July 2006
  \citembullet ``Tendon Growth and Healing: The Roles of Reaction,
  Transport and Mechanics,'' 15$^\mathrm{th}$ U.S. National Congress on
  Theoretical and Applied Mechanics, June 2006 
  \citembullet ``Computational Modelling of Mechanics and Transport in
  Growing Tissue,'' Eighth U.S. National Congress on Computational
  Mechanics, July 2005
  \citembullet ``A Continuum Treatment of Coupled Mass Transport and
  Mechanics in Growing Soft Tissue,'' Materials Research Society 2004
  Fall Meeting, November 2004
  \citembullet ``Multi-Scale Simulations of the Mechanics of Transport
  and Growth in Soft Tissue,'' 41$^\mathrm{st}$ Annual Technical
  Meeting of the Society of Engineering Science, October 2004 
  \citembullet ``Material forces in the context of biological tissue
  remodelling,'' Seventh U.S. National Congress on Computational
  Mechanics, July 2003
  \citembullet ``A continuum treatment of growth in tissue: Mass
  transport coupled with mechanics,'' Second M.I.T. Conference on
  Computational Fluid and Solid Mechanics, June 2003
\end{category}

%Activities that help with my work
\begin{category}{Professional Development}
  \citembullet Recognised as an {\em Engineering Academic Scholar} by
  the office of the Associate Dean for Graduate Education after
  successfully completing the {\em Academic Careers in Engineering and
    Sciences} program, November 2005
\end{category}

%Work-related service
\begin{category}{Professional Activities}
  \citembullet Engineering Graduate Student Symposium (2003 -- 2006):
  As an active member of the team responsible for organisation of the
  annual event, I have contributed to the growth of the symposium from
  its origins in the Department of Mechanical Engineering to now
  encompass most of the College of Engineering. The event serves as a
  lively forum for sharing research and fosters collaboration across
  departments. My roles have included chairing technical sessions,
  creating the symposium web site, photographing the event and
  creating its visual identity.
  \citembullet System administrator for the Computational Mechanics
  Laboratory (January 2003 -- Present) 
  \citembullet Writing articles related to free software and
  technology
\end{category}

\end{document}

